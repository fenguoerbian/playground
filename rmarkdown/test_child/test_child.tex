% Options for packages loaded elsewhere
\PassOptionsToPackage{unicode}{hyperref}
\PassOptionsToPackage{hyphens}{url}
%
\documentclass[
]{ctexart}
\usepackage{amsmath,amssymb}
\usepackage{lmodern}
\usepackage{iftex}
\ifPDFTeX
  \usepackage[T1]{fontenc}
  \usepackage[utf8]{inputenc}
  \usepackage{textcomp} % provide euro and other symbols
\else % if luatex or xetex
  \usepackage{unicode-math}
  \defaultfontfeatures{Scale=MatchLowercase}
  \defaultfontfeatures[\rmfamily]{Ligatures=TeX,Scale=1}
\fi
% Use upquote if available, for straight quotes in verbatim environments
\IfFileExists{upquote.sty}{\usepackage{upquote}}{}
\IfFileExists{microtype.sty}{% use microtype if available
  \usepackage[]{microtype}
  \UseMicrotypeSet[protrusion]{basicmath} % disable protrusion for tt fonts
}{}
\makeatletter
\@ifundefined{KOMAClassName}{% if non-KOMA class
  \IfFileExists{parskip.sty}{%
    \usepackage{parskip}
  }{% else
    \setlength{\parindent}{0pt}
    \setlength{\parskip}{6pt plus 2pt minus 1pt}}
}{% if KOMA class
  \KOMAoptions{parskip=half}}
\makeatother
\usepackage{xcolor}
\usepackage{color}
\usepackage{fancyvrb}
\newcommand{\VerbBar}{|}
\newcommand{\VERB}{\Verb[commandchars=\\\{\}]}
\DefineVerbatimEnvironment{Highlighting}{Verbatim}{commandchars=\\\{\}}
% Add ',fontsize=\small' for more characters per line
\usepackage{framed}
\definecolor{shadecolor}{RGB}{248,248,248}
\newenvironment{Shaded}{\begin{snugshade}}{\end{snugshade}}
\newcommand{\AlertTok}[1]{\textcolor[rgb]{0.94,0.16,0.16}{#1}}
\newcommand{\AnnotationTok}[1]{\textcolor[rgb]{0.56,0.35,0.01}{\textbf{\textit{#1}}}}
\newcommand{\AttributeTok}[1]{\textcolor[rgb]{0.77,0.63,0.00}{#1}}
\newcommand{\BaseNTok}[1]{\textcolor[rgb]{0.00,0.00,0.81}{#1}}
\newcommand{\BuiltInTok}[1]{#1}
\newcommand{\CharTok}[1]{\textcolor[rgb]{0.31,0.60,0.02}{#1}}
\newcommand{\CommentTok}[1]{\textcolor[rgb]{0.56,0.35,0.01}{\textit{#1}}}
\newcommand{\CommentVarTok}[1]{\textcolor[rgb]{0.56,0.35,0.01}{\textbf{\textit{#1}}}}
\newcommand{\ConstantTok}[1]{\textcolor[rgb]{0.00,0.00,0.00}{#1}}
\newcommand{\ControlFlowTok}[1]{\textcolor[rgb]{0.13,0.29,0.53}{\textbf{#1}}}
\newcommand{\DataTypeTok}[1]{\textcolor[rgb]{0.13,0.29,0.53}{#1}}
\newcommand{\DecValTok}[1]{\textcolor[rgb]{0.00,0.00,0.81}{#1}}
\newcommand{\DocumentationTok}[1]{\textcolor[rgb]{0.56,0.35,0.01}{\textbf{\textit{#1}}}}
\newcommand{\ErrorTok}[1]{\textcolor[rgb]{0.64,0.00,0.00}{\textbf{#1}}}
\newcommand{\ExtensionTok}[1]{#1}
\newcommand{\FloatTok}[1]{\textcolor[rgb]{0.00,0.00,0.81}{#1}}
\newcommand{\FunctionTok}[1]{\textcolor[rgb]{0.00,0.00,0.00}{#1}}
\newcommand{\ImportTok}[1]{#1}
\newcommand{\InformationTok}[1]{\textcolor[rgb]{0.56,0.35,0.01}{\textbf{\textit{#1}}}}
\newcommand{\KeywordTok}[1]{\textcolor[rgb]{0.13,0.29,0.53}{\textbf{#1}}}
\newcommand{\NormalTok}[1]{#1}
\newcommand{\OperatorTok}[1]{\textcolor[rgb]{0.81,0.36,0.00}{\textbf{#1}}}
\newcommand{\OtherTok}[1]{\textcolor[rgb]{0.56,0.35,0.01}{#1}}
\newcommand{\PreprocessorTok}[1]{\textcolor[rgb]{0.56,0.35,0.01}{\textit{#1}}}
\newcommand{\RegionMarkerTok}[1]{#1}
\newcommand{\SpecialCharTok}[1]{\textcolor[rgb]{0.00,0.00,0.00}{#1}}
\newcommand{\SpecialStringTok}[1]{\textcolor[rgb]{0.31,0.60,0.02}{#1}}
\newcommand{\StringTok}[1]{\textcolor[rgb]{0.31,0.60,0.02}{#1}}
\newcommand{\VariableTok}[1]{\textcolor[rgb]{0.00,0.00,0.00}{#1}}
\newcommand{\VerbatimStringTok}[1]{\textcolor[rgb]{0.31,0.60,0.02}{#1}}
\newcommand{\WarningTok}[1]{\textcolor[rgb]{0.56,0.35,0.01}{\textbf{\textit{#1}}}}
\usepackage{graphicx}
\makeatletter
\def\maxwidth{\ifdim\Gin@nat@width>\linewidth\linewidth\else\Gin@nat@width\fi}
\def\maxheight{\ifdim\Gin@nat@height>\textheight\textheight\else\Gin@nat@height\fi}
\makeatother
% Scale images if necessary, so that they will not overflow the page
% margins by default, and it is still possible to overwrite the defaults
% using explicit options in \includegraphics[width, height, ...]{}
\setkeys{Gin}{width=\maxwidth,height=\maxheight,keepaspectratio}
% Set default figure placement to htbp
\makeatletter
\def\fps@figure{htbp}
\makeatother
\setlength{\emergencystretch}{3em} % prevent overfull lines
\providecommand{\tightlist}{%
  \setlength{\itemsep}{0pt}\setlength{\parskip}{0pt}}
\setcounter{secnumdepth}{-\maxdimen} % remove section numbering
\ifLuaTeX
  \usepackage{selnolig}  % disable illegal ligatures
\fi
\IfFileExists{bookmark.sty}{\usepackage{bookmark}}{\usepackage{hyperref}}
\IfFileExists{xurl.sty}{\usepackage{xurl}}{} % add URL line breaks if available
\urlstyle{same} % disable monospaced font for URLs
\hypersetup{
  pdftitle={Untitled},
  pdfauthor={Chao Cheng},
  hidelinks,
  pdfcreator={LaTeX via pandoc}}

\title{Untitled}
\author{Chao Cheng}
\date{2022-09-21}

\begin{document}
\maketitle

{
\setcounter{tocdepth}{2}
\tableofcontents
}
\hypertarget{r-markdown}{%
\subsection{R Markdown}\label{r-markdown}}

This is an R Markdown document. Markdown is a simple formatting syntax
for authoring HTML, PDF, and MS Word documents. For more details on
using R Markdown see \url{http://rmarkdown.rstudio.com}.

When you click the \textbf{Knit} button a document will be generated
that includes both content as well as the output of any embedded R code
chunks within the document. You can embed an R code chunk like this:

\begin{Shaded}
\begin{Highlighting}[]
\FunctionTok{summary}\NormalTok{(cars)}
\end{Highlighting}
\end{Shaded}

\begin{verbatim}
##      speed           dist       
##  Min.   : 4.0   Min.   :  2.00  
##  1st Qu.:12.0   1st Qu.: 26.00  
##  Median :15.0   Median : 36.00  
##  Mean   :15.4   Mean   : 42.98  
##  3rd Qu.:19.0   3rd Qu.: 56.00  
##  Max.   :25.0   Max.   :120.00
\end{verbatim}

\hypertarget{including-plots}{%
\subsection{Including Plots}\label{including-plots}}

You can also embed plots, for example:

\includegraphics{test_child_files/figure-latex/pressure-1.pdf}

Note that the \texttt{echo\ =\ FALSE} parameter was added to the code
chunk to prevent printing of the R code that generated the plot.

\begin{Shaded}
\begin{Highlighting}[]
\NormalTok{knitr}\SpecialCharTok{::}\FunctionTok{all\_labels}\NormalTok{()}
\end{Highlighting}
\end{Shaded}

\begin{verbatim}
##  [1] "setup"           "cars"            "pressure"        "all_lab1"       
##  [5] "load_child"      "all_lab2"        "unnamed-chunk-1" "show_child"     
##  [9] "all_lab3"        "all_lab4"
\end{verbatim}

\begin{Shaded}
\begin{Highlighting}[]
\FunctionTok{ls}\NormalTok{()}
\end{Highlighting}
\end{Shaded}

\begin{verbatim}
## character(0)
\end{verbatim}

\begin{Shaded}
\begin{Highlighting}[]
\NormalTok{child\_res }\OtherTok{\textless{}{-}} \FunctionTok{lapply}\NormalTok{(}\FunctionTok{c}\NormalTok{(}\StringTok{"gen\_str\_output.Rmd"}\NormalTok{, }\StringTok{"form\_output.Rmd"}\NormalTok{),}
\NormalTok{                    knitr}\SpecialCharTok{::}\NormalTok{knit\_child, }
                    \AttributeTok{quiet =} \ConstantTok{TRUE}\NormalTok{, }
                    \AttributeTok{envir =} \FunctionTok{environment}\NormalTok{())}
\end{Highlighting}
\end{Shaded}

\begin{Shaded}
\begin{Highlighting}[]
\NormalTok{knitr}\SpecialCharTok{::}\FunctionTok{all\_labels}\NormalTok{()}
\end{Highlighting}
\end{Shaded}

\begin{verbatim}
##  [1] "setup"              "cars"               "pressure"          
##  [4] "all_lab1"           "load_child"         "all_lab2"          
##  [7] "unnamed-chunk-1"    "show_child"         "all_lab3"          
## [10] "all_lab4"           "fun_gen_str_output" "fun_form_output"
\end{verbatim}

\begin{Shaded}
\begin{Highlighting}[]
\FunctionTok{ls}\NormalTok{()}
\end{Highlighting}
\end{Shaded}

\begin{verbatim}
## [1] "child_res"           "Form_Output"         "Gen_Str_Output"     
## [4] "Gen_Str_Output_Atom"
\end{verbatim}

\begin{Shaded}
\begin{Highlighting}[]
\FunctionTok{Gen\_Str\_Output}\NormalTok{(}\DecValTok{1} \SpecialCharTok{:} \DecValTok{10}\NormalTok{, }\AttributeTok{digit =} \DecValTok{2}\NormalTok{)}
\end{Highlighting}
\end{Shaded}

\begin{verbatim}
##  [1] "1.00"  "2.00"  "3.00"  "4.00"  "5.00"  "6.00"  "7.00"  "8.00"  "9.00" 
## [10] "10.00"
\end{verbatim}

\hypertarget{appendix}{%
\section{appendix}\label{appendix}}

\hypertarget{details-of-predefined-functions}{%
\subsection{Details of predefined
functions}\label{details-of-predefined-functions}}

\begin{center}\rule{0.5\linewidth}{0.5pt}\end{center}

\texttt{Gen\_Str\_Output} 用来产生数值结果的字符串:

\begin{itemize}
\item
  \texttt{in\_num}: 数值型向量
\item
  \texttt{digit}:输出结果的小数点位数
\item
  \texttt{pct}:输入的数值是否是百分数,默认为\texttt{TRUE}。当输入数值代表百分数时,\texttt{0}和\texttt{100}
  将会直接输出,而不添加小数点和小数位数。
\end{itemize}

\begin{Shaded}
\begin{Highlighting}[]
\NormalTok{Gen\_Str\_Output\_Atom }\OtherTok{\textless{}{-}} \ControlFlowTok{function}\NormalTok{(in\_num, }\AttributeTok{digit =} \DecValTok{1}\NormalTok{, }\AttributeTok{pct =} \ConstantTok{TRUE}\NormalTok{)\{}
    \CommentTok{\# Generate string output from numeric input}
\NormalTok{    digit }\OtherTok{\textless{}{-}} \FunctionTok{min}\NormalTok{(digit, }\DecValTok{4}\NormalTok{)}
    
    \ControlFlowTok{if}\NormalTok{(pct)\{ }\CommentTok{\# rule for \textasciigrave{}percentage\textasciigrave{} output}
        \ControlFlowTok{if}\NormalTok{((in\_num }\SpecialCharTok{==} \DecValTok{0}\NormalTok{) }\SpecialCharTok{|}\NormalTok{ (in\_num }\SpecialCharTok{==} \DecValTok{100}\NormalTok{))\{}
\NormalTok{            res }\OtherTok{\textless{}{-}} \FunctionTok{as.character}\NormalTok{(in\_num)}
\NormalTok{        \}}\ControlFlowTok{else}\NormalTok{\{}
\NormalTok{            res }\OtherTok{\textless{}{-}} \FunctionTok{sprintf}\NormalTok{(}\FunctionTok{paste0}\NormalTok{(}\StringTok{"\%."}\NormalTok{, digit, }\StringTok{"f"}\NormalTok{), in\_num)}
\NormalTok{        \}}
\NormalTok{    \}}\ControlFlowTok{else}\NormalTok{\{ }\CommentTok{\# rule for other output}
\NormalTok{        res }\OtherTok{\textless{}{-}} \FunctionTok{sprintf}\NormalTok{(}\FunctionTok{paste0}\NormalTok{(}\StringTok{"\%."}\NormalTok{, digit, }\StringTok{"f"}\NormalTok{), in\_num)}
\NormalTok{    \}}
    \FunctionTok{return}\NormalTok{(res)}
\NormalTok{\}}

\NormalTok{Gen\_Str\_Output }\OtherTok{\textless{}{-}} \ControlFlowTok{function}\NormalTok{(in\_num, }\AttributeTok{digit =} \DecValTok{1}\NormalTok{, }\AttributeTok{pct =} \ConstantTok{TRUE}\NormalTok{)\{}
\NormalTok{    res }\OtherTok{\textless{}{-}} \FunctionTok{mapply}\NormalTok{(Gen\_Str\_Output\_Atom, }\AttributeTok{in\_num =}\NormalTok{ in\_num, }\AttributeTok{digit =}\NormalTok{ digit, }\AttributeTok{pct =}\NormalTok{ pct)}
    \FunctionTok{return}\NormalTok{(res)}
\NormalTok{\}}
\end{Highlighting}
\end{Shaded}

\begin{center}\rule{0.5\linewidth}{0.5pt}\end{center}

\texttt{Form\_Output} 将总结好的结果转换为TFL中要求的按列呈现的形式

\begin{itemize}
\item
  \texttt{df\_long}:
  待输出的数据,可参考\texttt{Summary\_Perct}的结果,主要需包含
\item
  \texttt{by\_var\_name}:提供\texttt{pivot\_wider}时的\texttt{names\_from}。其内容一般是剂量组(字符串或factor),在最终结果表中是列名(A组、B组\ldots\ldots)
\item
  \texttt{col\_name}: 该列保存计数结果
\item
  \texttt{\{col\_name\}\_pct\_str}:
  格式处理过后的百分比数值,(字符串格式)。
\item
  \texttt{group\_var\_name}:
  若非空,说明\texttt{df\_long}中数据是按照\texttt{(by\_var\_name,\ group\_var\_name)}这样的双层结构进行计数的。一般该列内容是各分组结果,如原因1,原因2,\ldots\ldots{}
\item
  \texttt{by\_var\_name}, \texttt{col\_name},
  \texttt{\{col\_name\}\_pct\_str}:已在之前解释
\end{itemize}

\begin{Shaded}
\begin{Highlighting}[]
\NormalTok{Form\_Output }\OtherTok{\textless{}{-}} \ControlFlowTok{function}\NormalTok{(df\_long, }
                        \AttributeTok{by\_var\_name =} \StringTok{"arm\_fct"}\NormalTok{, }
                        \AttributeTok{col\_name =} \StringTok{"trt\_num"}\NormalTok{, }
                        \AttributeTok{group\_var\_name =} \ConstantTok{NULL}\NormalTok{, }
                        \AttributeTok{out\_1st\_name =} \ConstantTok{NULL}\NormalTok{, }
                        \AttributeTok{out\_1st\_val =} \ConstantTok{NULL}\NormalTok{)\{}
\NormalTok{    res }\OtherTok{\textless{}{-}}\NormalTok{ df\_long }\SpecialCharTok{\%\textgreater{}\%}
        \FunctionTok{mutate}\NormalTok{(}\AttributeTok{out\_str =} \FunctionTok{str\_c}\NormalTok{(.data[[col\_name]], }
                               \StringTok{"("}\NormalTok{, }
\NormalTok{                               .data[[glue}\SpecialCharTok{::}\FunctionTok{glue}\NormalTok{(}\StringTok{"\{var\_name\}\_pct\_str"}\NormalTok{, }
                                                 \AttributeTok{var\_name =}\NormalTok{ col\_name)]], }
                               \StringTok{"\%)"}\NormalTok{))}
    \ControlFlowTok{if}\NormalTok{(}\FunctionTok{is.null}\NormalTok{(group\_var\_name))\{}
\NormalTok{        res }\OtherTok{\textless{}{-}}\NormalTok{ res }\SpecialCharTok{\%\textgreater{}\%}
            \FunctionTok{select}\NormalTok{(}\FunctionTok{all\_of}\NormalTok{(by\_var\_name), out\_str) }\SpecialCharTok{\%\textgreater{}\%}
            \FunctionTok{pivot\_wider}\NormalTok{(}\AttributeTok{names\_from =} \FunctionTok{all\_of}\NormalTok{(by\_var\_name), }
                        \AttributeTok{values\_from =}\NormalTok{ out\_str) }
        \ControlFlowTok{if}\NormalTok{(}\SpecialCharTok{!}\FunctionTok{is.null}\NormalTok{(out\_1st\_name))\{}
\NormalTok{            res }\OtherTok{\textless{}{-}}\NormalTok{ res }\SpecialCharTok{\%\textgreater{}\%}
                \FunctionTok{mutate}\NormalTok{(}\StringTok{"\{out\_1st\_name\}"} \SpecialCharTok{:}\ErrorTok{=}\NormalTok{ out\_1st\_val, }\AttributeTok{.before =} \DecValTok{1}\NormalTok{)}
\NormalTok{        \}}\ControlFlowTok{else}\NormalTok{\{}
\NormalTok{            res }\OtherTok{\textless{}{-}}\NormalTok{ res }\SpecialCharTok{\%\textgreater{}\%}
                \FunctionTok{mutate}\NormalTok{(}\StringTok{"\{col\_name\}"} \SpecialCharTok{:}\ErrorTok{=} \StringTok{" "}\NormalTok{, }\AttributeTok{.before =} \DecValTok{1}\NormalTok{)}
\NormalTok{        \}}
\NormalTok{    \}}\ControlFlowTok{else}\NormalTok{\{}
\NormalTok{        grp\_lvls }\OtherTok{\textless{}{-}} \FunctionTok{levels}\NormalTok{(df\_long }\SpecialCharTok{\%\textgreater{}\%} \FunctionTok{pull}\NormalTok{(}\FunctionTok{all\_of}\NormalTok{(group\_var\_name)))}
        
\NormalTok{        res }\OtherTok{\textless{}{-}}\NormalTok{ res }\SpecialCharTok{\%\textgreater{}\%}
            \FunctionTok{pivot\_wider}\NormalTok{(}\AttributeTok{id\_cols =}\NormalTok{ .data[[group\_var\_name]], }
                        \AttributeTok{names\_from =} \FunctionTok{all\_of}\NormalTok{(by\_var\_name), }
                        \AttributeTok{values\_from =}\NormalTok{ out\_str) }\SpecialCharTok{\%\textgreater{}\%}
            \FunctionTok{arrange}\NormalTok{(}\FunctionTok{factor}\NormalTok{(.data[[group\_var\_name]], }\AttributeTok{levels =}\NormalTok{ grp\_lvls)) }\SpecialCharTok{\%\textgreater{}\%}    \CommentTok{\# 确保输出行的顺序与原始\textasciigrave{}group\_var\_name\textasciigrave{}的level一致}
            \FunctionTok{mutate}\NormalTok{(}\StringTok{"\{group\_var\_name\}"} \SpecialCharTok{:}\ErrorTok{=} \FunctionTok{as.character}\NormalTok{(.data[[group\_var\_name]]))}
        \ControlFlowTok{if}\NormalTok{(}\SpecialCharTok{!}\FunctionTok{is.null}\NormalTok{(out\_1st\_name))\{}
\NormalTok{            res }\OtherTok{\textless{}{-}}\NormalTok{ res }\SpecialCharTok{\%\textgreater{}\%}
                \FunctionTok{rename}\NormalTok{(}\StringTok{"\{out\_1st\_name\}"} \SpecialCharTok{:}\ErrorTok{=} \FunctionTok{all\_of}\NormalTok{(group\_var\_name))}
\NormalTok{        \}}
\NormalTok{    \}}
    
    \FunctionTok{return}\NormalTok{(res)}
\NormalTok{\}}
\end{Highlighting}
\end{Shaded}

\begin{Shaded}
\begin{Highlighting}[]
\NormalTok{knitr}\SpecialCharTok{::}\FunctionTok{all\_labels}\NormalTok{()}
\end{Highlighting}
\end{Shaded}

\begin{verbatim}
##  [1] "setup"              "cars"               "pressure"          
##  [4] "all_lab1"           "load_child"         "all_lab2"          
##  [7] "unnamed-chunk-1"    "show_child"         "all_lab3"          
## [10] "all_lab4"           "fun_gen_str_output" "fun_form_output"
\end{verbatim}

\begin{Shaded}
\begin{Highlighting}[]
\NormalTok{Gen\_Str\_Output\_Atom }\OtherTok{\textless{}{-}} \ControlFlowTok{function}\NormalTok{(in\_num, }\AttributeTok{digit =} \DecValTok{1}\NormalTok{, }\AttributeTok{pct =} \ConstantTok{TRUE}\NormalTok{)\{}
    \CommentTok{\# Generate string output from numeric input}
\NormalTok{    digit }\OtherTok{\textless{}{-}} \FunctionTok{min}\NormalTok{(digit, }\DecValTok{4}\NormalTok{)}
    
    \ControlFlowTok{if}\NormalTok{(pct)\{ }\CommentTok{\# rule for \textasciigrave{}percentage\textasciigrave{} output}
        \ControlFlowTok{if}\NormalTok{((in\_num }\SpecialCharTok{==} \DecValTok{0}\NormalTok{) }\SpecialCharTok{|}\NormalTok{ (in\_num }\SpecialCharTok{==} \DecValTok{100}\NormalTok{))\{}
\NormalTok{            res }\OtherTok{\textless{}{-}} \FunctionTok{as.character}\NormalTok{(in\_num)}
\NormalTok{        \}}\ControlFlowTok{else}\NormalTok{\{}
\NormalTok{            res }\OtherTok{\textless{}{-}} \FunctionTok{sprintf}\NormalTok{(}\FunctionTok{paste0}\NormalTok{(}\StringTok{"\%."}\NormalTok{, digit, }\StringTok{"f"}\NormalTok{), in\_num)}
\NormalTok{        \}}
\NormalTok{    \}}\ControlFlowTok{else}\NormalTok{\{ }\CommentTok{\# rule for other output}
\NormalTok{        res }\OtherTok{\textless{}{-}} \FunctionTok{sprintf}\NormalTok{(}\FunctionTok{paste0}\NormalTok{(}\StringTok{"\%."}\NormalTok{, digit, }\StringTok{"f"}\NormalTok{), in\_num)}
\NormalTok{    \}}
    \FunctionTok{return}\NormalTok{(res)}
\NormalTok{\}}

\NormalTok{Gen\_Str\_Output }\OtherTok{\textless{}{-}} \ControlFlowTok{function}\NormalTok{(in\_num, }\AttributeTok{digit =} \DecValTok{1}\NormalTok{, }\AttributeTok{pct =} \ConstantTok{TRUE}\NormalTok{)\{}
\NormalTok{    res }\OtherTok{\textless{}{-}} \FunctionTok{mapply}\NormalTok{(Gen\_Str\_Output\_Atom, }\AttributeTok{in\_num =}\NormalTok{ in\_num, }\AttributeTok{digit =}\NormalTok{ digit, }\AttributeTok{pct =}\NormalTok{ pct)}
    \FunctionTok{return}\NormalTok{(res)}
\NormalTok{\}}
\end{Highlighting}
\end{Shaded}

\begin{Shaded}
\begin{Highlighting}[]
\NormalTok{knitr}\SpecialCharTok{::}\FunctionTok{all\_labels}\NormalTok{()}
\end{Highlighting}
\end{Shaded}

\begin{verbatim}
##  [1] "setup"              "cars"               "pressure"          
##  [4] "all_lab1"           "load_child"         "all_lab2"          
##  [7] "unnamed-chunk-1"    "show_child"         "all_lab3"          
## [10] "all_lab4"           "fun_gen_str_output" "fun_form_output"
\end{verbatim}

\end{document}
